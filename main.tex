\documentclass[12pt, a4paper, twoside]{article}

%% Preamble
\usepackage{umatfgspanish}
\usepackage{blindtext}
\usepackage{biblatex}
\usepackage{csquotes}
\usepackage[spanish]{babel}
\usepackage{listings}
\usepackage{array}
\usepackage{float}
\usepackage{graphicx}
\usepackage{xurl}

\addbibresource{references.bib}

  
\graphicspath{{images/}{../images/}}


\definecolor{codegreen}{rgb}{0,0.6,0}
\definecolor{codegray}{rgb}{0.5,0.5,0.5}
\definecolor{codepurple}{rgb}{0.58,0,0.82}
\definecolor{backcolour}{rgb}{0.95,0.95,0.95}

\lstdefinestyle{mystyle}{
    backgroundcolor=\color{backcolour},   % choose the background color
    basicstyle=\footnotesize,        % the size of the fonts that are used for the code
    breakatwhitespace=false,         % sets if automatic breaks should only happen at whitespace
    breaklines=true,                 % sets automatic line breaking
    captionpos=t,                    % sets the caption-position to top
    commentstyle=\color{codegreen},    % comment style
    extendedchars=true,              % lets you use non-ASCII characters; for 8-bits encodings only, does not work with UTF-8
    firstnumber=1,                % start line enumeration with line 1000
    frame=top-bottom,                    % adds a frame around the code
    keepspaces=true,                 % keeps spaces in text, useful for keeping indentation of code (possibly needs columns=flexible)
    keywordstyle=\color{codepurple},       % keyword style
    morekeywords={*,...},            % if you want to add more keywords to the set
    numbers=left,                    % where to put the line-numbers; possible values are (none, left, right)
    numbersep=5pt,                   % how far the line-numbers are from the code
    numberstyle=\tiny\color{mygray}, % the style that is used for the line-numbers
    rulecolor=\color{black},         % if not set, the frame-color may be changed on line-breaks within not-black text (e.g. comments (green here))
    showspaces=false,                % show spaces everywhere adding particular underscores; it overrides 'showstringspaces'
    showstringspaces=false,          % underline spaces within strings only
    showtabs=false,                  % show tabs within strings adding particular underscores
    stepnumber=1,                    % the step between two line-numbers. If it's 1, each line will be numbered
    stringstyle=\color{codegray},     % string literal style
    tabsize=2,                       % sets default tabsize to 2 spaces
}

\lstset{style=mystyle}
\renewcommand{\lstlistingname}{Código}

\usepackage{subfiles} % Best loaded last in the preamble

\begin{document}

%% Cover
% \includepdf[noautoscale=true, width=\paperwidth]{cover.pdf}

%% Title
\clearpage
\setcounter{page}{1}

% \includepdf[noautoscale=true, width=\paperwidth]{title.pdf}
\includepdf[noautoscale=true, width=\paperwidth]{PortadaTFM.pdf}

\newpage

\subfile{sections/resumen}

\newpage

%% Abstract
\subfile{sections/abstract}

\tableofcontents
\listoffigures

%% Sections

%% ==================== NOTAS ====================

%% - Model-agnostic methods (SHAP) vs Model-specific methods (LIME)

%% ===============================================

\section{Introducción}
\subfile{sections/introduccion}

% Que es el mlops
% Que es el drift
% De donde viene la idea de utilizar explicabilidad
% Motivacion general del proyecto
% Objetivos del proyecto

\section{Estado del arte}
\subfile{sections/estadoArte}
% Hablar sobre el incio del MLOPS
% Explicar las diferentes propuestas existentes para detectar data, concept y model drift
% Explicar las propuestas usando metodologias de explicabilidad

\section{Metodología y experimentación}
\subfile{sections/metodologia}
% Ofrecer una solución a la detección de model drift de forma generalista usando un modelo de explicabilidad
% Dar solucion al problema del drift
% Usar modelos de explicabilidad sobre un caso de uso dado

\section{Resultados}
\subfile{sections/resultados}
% Describir los resultados obtenidos
% Tablas gráficas y demas

\section{Conclusiones y líneas futuras de investigación}
\subfile{sections/conclusiones}
% Mostrar resultados
% Dar conclusiones
% Futuras líneas: Imágenes, inclusión en un workflow, reentrenamiento

%% Bibliography
\nocite{*}
\printbibliography

\newpage

%% Apendices
% \begin{umaappendices}
%     \section{Guía de instalación}
%     \subfile{sections/appendixA}
% \end{umaappendices}

%% Back Cover
\includepdf[noautoscale=true, width=\paperwidth]{backcover.pdf}
\end{document}
